\documentclass[mathserif,final]{beamer} %
%\documentclass[final]{beamer} %

\usepackage[english]{babel}
%\usepackage[latin1]{inputenc}
\usepackage[ansinew]{inputenc}
\usepackage{amsmath,amsthm, amssymb,latexsym}
\usepackage{graphicx}
\usepackage{setspace}

\usepackage[T1]{fontenc}
%\renewcommand{\sfdefault}{lhv}
%\usepackage[euler-hat-accent]{eulervm}

\usepackage[small,vertical,titleright]{posterlti}

% -------------------- Definitionen --------------------
\def\d{ \mathrm{d} }											% integral d
\def\I{ \mathrm{i} }											% imaginary unit
\def\E{ \mathrm{e} }											% e
\def\uC{ \mathbb{T} }											% unit circle
\def\uD{ \mathbb{D} }											% unit disk

\def\H{ \mathcal{H} }											% Hilbert space

\def\RN{ \mathbb{R} }											% real numbers
\def\CN{ \mathbb{C} }											% complex numbers
\def\ZN{ \mathbb{Z} }											% integers
\def\NN{ \mathbb{N} }											% non-negative integers

\def\bss{ \boldsymbol{s} }								% Sequence s
\def\bssigma{ \boldsymbol{\sigma} }				% Sequence sigma

\def\Tr{ \mathrm{S} }											% translation operator
\def\Pr{ \mathrm{P} }											% projection

\DeclareMathOperator*{\cls}{\overline{span}}			% closed linear span
\DeclareMathOperator*{\argmin}{arg\, min}					% argmin
% ------------------------------------------------------




\title{\huge Causal Reconstruction Kernels for Consistent Signal Recovery}
\author{\bfseries Volker Pohl, Fanny Yang and Holger Boche}
\institute{Lehrstuhl f�r Theoretische Informationstechnik, Technische Universit\"at M\"unchen}
\date{August 30$^{th}$, 2012}
\setfootlinelti{EUSIPCO 2012 -- Session: Signal Processing Detection and Estimation}{\{volker.pohl, fanny.yang, boche\}@tum.de}




\begin{document}
\begin{frame}
\begin{columns} 
%%%%%%%%%%%%%%%%%%%%%%%%%%%%%%%%%%%%%%%%%%%%%%%%%%%%%%%%%%%%%%%%%%%%%%%%%% 
%%%%%%%%%%%%%%%%%%%%%%%%%%%%%%%%%%%%%%%%%%%%%%%%%%%%%%%%%%%%%%%%%%%%%%%%%% 
% COLUMN 1
%%%%%%%%%%%%%%%%%%%%%%%%%%%%%%%%%%%%%%%%%%%%%%%%%%%%%%%%%%%%%%%%%%%%%%%%%% 
%%%%%%%%%%%%%%%%%%%%%%%%%%%%%%%%%%%%%%%%%%%%%%%%%%%%%%%%%%%%%%%%%%%%%%%%%%
\begin{column1lti}


% ============================== Motivation ==============================
\begin{blocklti}{Motivation -- Causal Reconstruction}

Causal signal reconstruction is crucial
\begin{itemize}
\item in on-line applications
\item feedback loops in control systems
\item to reduce border effects in image processing, etc.
\end{itemize}
\end{blocklti}
% ========================================================================






% ============================== U-Invariant Sampling ==============================
\begin{blocklti}{Framework -- Shift Invariant Sampling}

\begin{columns}[t]
% --- linke Spalte ---
\begin{column}{0.5\linewidth}
\centering
\structure{\bfseries Non-ideal Acquisition}

\begin{figure}[t]
\begin{center}
\begin{picture}(350,100)(50,70)
		\thicklines
    \put(50,120){\makebox(0,0){\footnotesize $x(t)$}}
    \put(10,100){\vector(1,0){90}}
    % --- Filter ---
    \put(100,70){\line(1,0){80}}
    \put(100,130){\line(1,0){80}}
    \put(100,70){\line(0,1){60}}
    \put(180,70){\line(0,1){60}}
    \put(140,150){\makebox(0,0){\footnotesize lowpass filter}}
    \put(140,100){\makebox(0,0){\footnotesize $h(t)$}}
		\put(180,100){\line(1,0){80}}
		\put(220,120){\makebox(0,0){\footnotesize $y(t)$}}
	  % --- Sampler ---
	  \put(260,100){\line(2,1){30}} %x=90
	  \qbezier(263,115)(278,113)(280,100)
	  \put(280,100){\vector(0,-1){5}}
	  \put(290,100){\vector(1,0){50}}
    \put(275,75){\makebox(0,0){\footnotesize $t = n a$}}
    % --- Output ---
    \put(350,120){\makebox(0,0){\footnotesize $c_{n} = y(n a)$}}
\end{picture}
\end{center}
\end{figure}

\begin{eqnarray*}
	c_{n} &=& y(n a) = \int^{\infty}_{-\infty} h(\tau)\, x( na - \tau)\, \d\tau
\end{eqnarray*}

\end{column}

% --- rechte Spalte ---
\begin{column}{0.5\linewidth}
\centering
\structure{\bfseries Shift-Invariant Reconstruction}

\begin{figure}[t]
\begin{center}
\begin{picture}(280,120)(-20,55)
		%\linethickness{0.75mm}
		\thicklines
		\put(-20,100){\vector(1,0){50}}
		\put(0,120){\makebox(0,0){\footnotesize $\{ c_{n} \}$}}
		\put(40,100){\circle{20}}
    \put(40,100){\makebox(0,0){$\times$}}
    \put(40,70){\vector(0,1){20}}
    \put(40,55){\makebox(0,0){\footnotesize $\sum_{n}\delta(t - n a)$}}
		%
    \put(50,100){\vector(1,0){50}}
    % --- Filter ---
    \put(100,70){\line(1,0){80}}
    \put(100,130){\line(1,0){80}}
    \put(100,70){\line(0,1){60}}
    \put(180,70){\line(0,1){60}}
    \put(140,150){\makebox(0,0){\footnotesize linear filter}}
    \put(140,100){\makebox(0,0){\footnotesize $\sigma(t)$}}
		\put(180,100){\vector(1,0){80}}
		\put(220,120){\makebox(0,0){\footnotesize $\widetilde{x}(t)$}}
\end{picture}
\end{center}
\end{figure}

\begin{eqnarray*}
	& \widetilde{x}(t) = \sum_{n\in\ZN} c_{n}\, \sigma(t - n a)
\end{eqnarray*}

\end{column}
%
\end{columns}

\vspace{1ex}

\begin{itemize}
\item Subordinate signal space: arbitrary Hilbert space $\H$
\item The sampling process can be described as the evaluation inner products
\begin{equation*}
	c_{n} = y(na) = \left\langle x, s_{n} \right\rangle\;,
	\qquad n\in\ZN
\end{equation*}
\item Sampling functions $s_{n} \in \H$ have the form $s_{n} = (\mathrm{T}^{n}_{a} s)(t) = s(t-na)$
\begin{flalign*}
	&\text{or in general} \qquad 
	s_{n} = \mathrm{U}^{n} s
	\quad \text{with}\quad
	\left\{
	\begin{array}{l}
		\text{generator}\ s\in\H\\
		\mathrm{U}\ \text{unitary operator on}\ \H
	\end{array}\right.&
\end{flalign*}
\item[$\Rightarrow$] Sequence $\bss = \{s_{n}\}_{n\in\ZN}$ of sampling functions forms a stationary sequence in $\H$
\item[$\Rightarrow$] Sequence $\bss$ is characterized by its corresponding spectral density $\Phi_{\bss} \in L^{1}(\uC)$
\begin{equation*}
	\left\langle s_{n}, s_{m} \right\rangle
	= \frac{1}{2\pi} \int^{\pi}_{-\pi} \E^{\I (n - m)\theta}\, \Phi_{\bss}(\E^{\I\theta})\, \d\theta
\end{equation*}

\item Sampling space: $\mathcal{S} := \cls\{ s_{n} : n\in\ZN \}$
\end{itemize}


% ----- Properties -----

\begin{itemize}
\item 

$\bss = \{s_{n}\}_{n\in\ZN}$ is a Riesz basis for the sampling space $\mathcal{S}$

\begin{tabular}{p{5.6cm}l}
 & $\Leftrightarrow\quad$ $0  < A \leq \Phi_{\bss}(\E^{\I\theta}) \leq B < \infty$ for a.e. $\theta \in [-\pi,\pi)$
\end{tabular}
\end{itemize}

\hrule

\vspace{2ex}

\begin{thebibliography}{1}
\begin{spacing}{2.0}
\bibitem{Unser_SP94}
{ \footnotesize M.~Unser and A.~Aldroubi, ``{A General Sampling Theory for Nonideal Acquisition
  Devices},'' \emph{{IEEE} Trans. Signal Process.}, vol.~42, no.~11, pp.
  2915--2925, Nov. 1994.}

\vspace{1ex}

\bibitem{MPE_IEEE_SP_11}
{ \footnotesize T.~Michaeli, V.~Pohl, and Y.~C. Eldar, ``{U-Invariant Sampling: Extrapolation
  and Causal Interpolation from Generalized Samples},'' \emph{{IEEE} Trans.
  Signal Process.}, vol.~59, no.~5, pp. 2085--2100, May 2011. }
\end{spacing}
\end{thebibliography}
\end{blocklti}
% ==================================================================================


\vfill

% ============================== Problem 1 ==============================
\begin{blocklti}{Reconstruction -- Ideal World}

\structure{\bfseries Assumptions}

\begin{itemize}
\item Let $x \in \mathcal{S}$ be an arbitrary signal
\item All past and future signal samples $c_{n} = \left\langle x, s_{n}\right\rangle$, $n\in \ZN$ are known
\end{itemize}

\vspace{1ex}
\structure{\bfseries Goal}

Signal reconstruction of the form
\begin{eqnarray*}
	& \widetilde{x}(t) = \sum_{n\in\ZN} \left\langle x,s_{n}\right\rangle \sigma_{n}(t)
\end{eqnarray*}
such that
\begin{itemize}
\item $\widetilde{x}(t) = x(t)$ for all $x \in \mathcal{S}$ (Perfect reconstruction)
\item $\left\langle \widetilde{x},s_{n} \right\rangle = \left\langle x,s_{n}\right\rangle$ for all $n \in \ZN$ (Consistency)

\end{itemize}

\vspace{1ex}
\structure{\bfseries Solution}

A well known result from frame theory states that the problem is solved by the dual Riesz basis $\{ \sigma_{n} \}_{n\in\ZN}$
of $\{s_{n}\}_{n\in\ZN}$, given by

\vspace{-2ex}

\begin{gather}
\label{equ:ncDual}
  \sigma_{n}(t) = (\mathrm{U}^{n} \sigma)(t)\\
	\text{with} \quad \sigma(t) = \sum_{k\in\ZN} \alpha_{k}\, s_{-k}(t)
	\quad \text{and}\quad
	\alpha_{k} = \frac{1}{2\pi}\int^{\pi}_{-\pi} \frac{\E^{-\I k \theta}}{\Phi_{\bss}(\E^{\I\theta})}\, \d\theta.\nonumber
\end{gather}


\end{blocklti}
% =======================================================================================

\vfill

% ============================== PROBLEM =============================================
\begin{colorblocklti}[yellow!90!white]
\begin{center}
{\bfseries Often, in reality only the past signal samples are known!}
\end{center}

\end{colorblocklti}
% ====================================================================================

\vfill


% ============================== Causal Reconstruction ===============================
\begin{blocklti}{Reconstruction -- Real World}
\structure{\bfseries New Assumption}

Let $\boldsymbol{c}_{0} = \{ c_{0}, c_{-1}, \dots \}$ be the past signal samples known at $t = t_{0}$

\vspace{1ex}
\structure{\bfseries Goal}

%% Reconstruct the past signal component
%% \begin{equation*}
%% 	x_{-}(t)
%% 	= \left\{\begin{array}{ll}
%% 	x(t)\quad & \text{if}\ t < t_{0}\\
%% 	0\quad & \text{if}\ t \geq t_{0}
%% 	\end{array}\right.
%% \end{equation*}
%% of the signal $x(t)$ at time $t=t_{0}$ from the past signal samples $\boldsymbol{c}_{0}$ only.

\vspace{1ex}
\structure{\bfseries Naive Solution}
\begin{equation*}
	\widetilde{x}_{-}(t) = \sum^{\infty}_{n=0} c_{-n} \sigma_{-n}(t)\, ,
	\quad t\leq t_{0}
\end{equation*}
based on the non-causal dual frame \eqref{equ:ncDual}.

	


\end{blocklti}
% ====================================================================================


\end{column1lti}
%%%%%%%%%%%%%%%%%%%%%%%%%%%%%%%%%%%%%%%%%%%%%%%%%%%%%%%%%%%%%%%%%%%%%%%%%% 
%%%%%%%%%%%%%%%%%%%%%%%%%%%%%%%%%%%%%%%%%%%%%%%%%%%%%%%%%%%%%%%%%%%%%%%%%%
% column 2
%%%%%%%%%%%%%%%%%%%%%%%%%%%%%%%%%%%%%%%%%%%%%%%%%%%%%%%%%%%%%%%%%%%%%%%%%%
%%%%%%%%%%%%%%%%%%%%%%%%%%%%%%%%%%%%%%%%%%%%%%%%%%%%%%%%%%%%%%%%%%%%%%%%%%
\begin{column2lti}


% ============================== PROBLEM =============================================
\begin{colorblocklti}[yellow!90!white]
\begin{center}
{\bfseries 
However this reconstruction is not perfect, i.e. $\widetilde{x}_{-}(t) \neq x_{-}(t)$,\\
because we need the dual Riesz basis $\{ \zeta_{-n}\}^{\infty}_{n=0}$ of $\{ s_{-n} \}^{\infty}_{n=0}$ !}
\end{center}

\end{colorblocklti}
% ====================================================================================


\vfill

% ============================== Main Result ==============================
\begin{blockredlti}{Main Result -- Causal Dual Riesz Basis}

\structure{\bfseries Theorem}
Let $\bss = \{s_{n}\}_{n\in\ZN}$ be a stationary sequence in a Hilbert space $\H$ which is a Riesz basis for $\mathcal{S} = \cls\{ s_{n} : n\in\ZN\}$ and let $\Phi_{\bss}$ be
the spectral density of $\bss$.\\
Then $\bss_{0} = \{s_{-n}\}^{\infty}_{n=0}$ is a Riesz basis for $\mathcal{S}_{0} = \cls\{ s_{-n} : n=0,1,2,\dots\}$ and the corresponding dual Riesz basis $\{\zeta_{-n}\}^{\infty}_{n=0}$ is given by
\begin{equation*}
\label{equ:CausalDual}
	\zeta_{-n} = \sum^{\infty}_{k=0} \widehat{\psi}_{n}(k)\, s_{-k}\;,
	\quad n=0,1,2,\dots
\end{equation*}

\begin{equation*}
\label{equ:DualFourCoef}
	\text{with}\qquad
	\widehat{\psi}_{n}(k) = \frac{1}{2\pi} \int^{\pi}_{-\pi} \psi_{n}(\E^{\I\theta})\, \E^{-\I k \theta}\, \d\theta\;,
	\quad n=0,1,2,\dots
\end{equation*}
$\psi_{n} \in H^{2}$ are defined as
\begin{equation*}
\label{equ:CausalDualPsi}
	\psi_{n}(\E^{\I\theta})
	= \frac{1}{\Phi^{+}_{\bss}(\E^{\I\theta})}\, \Pr_{+}\left[ \frac{\E^{\I n \theta}}{\Phi^{-}_{\bss}(\E^{\I\theta})} \right]\;,
	\quad n = 0,1,2,\dots
\end{equation*}
and wherein $\Phi^{+}_{\bss}$ and $\Phi^{-}_{\bss}$ are the spectral factors of $\Phi_{\bss}$.

\vspace{1ex}
\hrule
\vspace{1ex}

\structure{\bfseries Overall Causal Reconstruction Scheme}

\begin{columns}[t]
 % --- linke Spalte ---
\begin{column}{0.4\linewidth}


\begin{figure}[t]
\begin{center}
\begin{picture}(350,280)(-50,40)
		%\linethickness{0.75mm}
		\thicklines
    \put(-10,320){\makebox(0,0){\footnotesize $\{c_{n}\}_{n\leq 0}$}}
    % --- Filter oben ---
    \put(-50,300){\vector(1,0){150}}
    \put(70,310){\makebox(0,0){\footnotesize $c_{0}$}}
    \put(100,270){\line(1,0){80}}
    \put(100,330){\line(1,0){80}}
    \put(100,270){\line(0,1){60}}
    \put(180,270){\line(0,1){60}}
    \put(140,300){\makebox(0,0){\footnotesize $\zeta_{0}(t)$}}
		\put(180,300){\line(1,0){57}}
    % --- Filter mitte ---
    \put(35,200){\vector(1,0){65}}
    \put(70,210){\makebox(0,0){\footnotesize $c_{-1}$}}
    \put(100,170){\line(1,0){80}}
    \put(100,230){\line(1,0){80}}
    \put(100,170){\line(0,1){60}}
    \put(180,170){\line(0,1){60}}
    \put(140,200){\makebox(0,0){\footnotesize $\zeta_{-1}(t)$}}
		\put(180,200){\line(1,0){70}}
    % --- Filter unten ---
    \put(35,100){\vector(1,0){65}}
    \put(70,110){\makebox(0,0){\footnotesize $c_{-2}$}}
    \put(100,70){\line(1,0){80}}
    \put(100,130){\line(1,0){80}}
    \put(100,70){\line(0,1){60}}
    \put(180,70){\line(0,1){60}}
    \put(140,100){\makebox(0,0){\footnotesize $\zeta_{-2}(t)$}}
		\put(180,100){\line(1,0){70}}
		% --- Shift oben ---
		\put(35,300){\line(0,-1){30}}
    \put(10,270){\line(1,0){50}}
    \put(10,230){\line(1,0){50}}
    \put(60,230){\line(0,1){40}}
    \put(10,230){\line(0,1){40}}
		\put(35,250){\makebox(0,0){\footnotesize $z^{-1}$}}
		% --- Shift mitte ---
		\put(35,230){\line(0,-1){60}}
    \put(10,170){\line(1,0){50}}
    \put(10,130){\line(1,0){50}}
    \put(60,130){\line(0,1){40}}
    \put(10,130){\line(0,1){40}}
		\put(35,150){\makebox(0,0){\footnotesize $z^{-1}$}}
		% --- punkt punkt punkt ---
		\put(35,130){\line(0,-1){60}}
		\put(35,50){\makebox(0,0){\vdots}}
		\put(140,50){\makebox(0,0){\vdots}}
		\put(250,50){\makebox(0,0){\vdots}}
		% --- Summe ---
		\put(250,300){\circle{26}}
    \put(250,300){\makebox(0,0){$+$}}
    \put(250,287){\line(0,-1){217}}
    \put(263,300){\vector(1,0){50}}
    \put(290,320){\makebox(0,0){\footnotesize $\widetilde{x}_{-}(t)$}}
\end{picture}
\end{center}
\end{figure}
\end{column}


% --- rechte Spalte ---
\begin{column}{0.6\linewidth}

\vspace{-1ex}


\begin{itemize}

\item Reconstructed past signal

\vspace{-2ex}

\begin{eqnarray*}
\label{equ:Reconstr}
	& \widetilde{x}_{-}(t)
	= \sum_{n\leq 0} c_{-n}\, \zeta_{-n}(t)\;,
	\quad t\leq t_{0}
\end{eqnarray*}


\item Each reconstruction kernel $\zeta_{-n}(t)$ has a different shape  
\item Each sampling value corresponds to the weighting of the respective kernel
\end{itemize}
\end{column}

\end{columns}


\end{blockredlti}
% ===================================================================================================================


\vfill


% ======================================== EXAMPLE - CAUSAL SPLINE INTERPOLATION ========================================
\begin{examplelti}[Example -- Causal Spline Reconstruction]

\structure{\bfseries Practical Assumptions}

\vspace{-2ex}

% ----- Bild Spline -----
\begin{columns}[t]
 % --- linke Spalte ---
 \begin{column}{0.5\linewidth}
 
 \begin{itemize}
 \item Shift-invariant sampling in $L^{2}(\RN)$ with period $a=1$: $(\mathrm{U}^ns)(t) = s(t-n)$
 \item Impulse response $s(t)$ is a B-spline of 2nd degree as a model of a non-ideal lowpass
 \end{itemize}
 \begin{figure}[t]
  \centering
  \includegraphics[width=150mm]{Gr_TP2.pdf}
 \end{figure}

 \end{column}

 % --- rechte Spalte ---
 \begin{column}{0.5\linewidth}
  \begin{figure}[t]
  \centering
  \includegraphics[width=182mm]{Gr_Spline22.pdf}
  \label{Generators}
 \end{figure}
 {\footnotesize $s(t)$: generator of the sampling functions\\[-0.5ex]
 $\sigma(t)$: generator of the non-causal dual basis}
 \end{column}

\end{columns}



% ---- Vergleich - Kerne -----
\vspace{1ex}
\hrule
\vspace{1ex}
\structure{\bfseries Causal versus Non-causal Reconstruction Kernels}

\begin{columns}[t]
 % --- linke Spalte ---
 \begin{column}{0.55\linewidth}
  \begin{itemize}
  \item Dashed lines: truncated non-causal reconstruction kernels
  \begin{equation*}
     \sigma_{-n}(t)
     = \sum_{k\in\ZN} \alpha_{k}\, s_{-k}(t+n)
  \end{equation*}
  \item Solid lines: causal reconstruction kernels
  \begin{equation*}
   \zeta_{-n}(t) = \sum^{\infty}_{k=0} \widehat{\psi}_{n}(k)\, s_{-k}(t)
  \end{equation*}
  \item For large $n$, the causal kernel $\zeta_{-n}(t)$ converges to non-causal kernel $\sigma_{-n}(t)$
  \begin{equation*}
  	\zeta_{-n} \to \sigma_{-n}
  	\quad\text{as}\quad
  	n \to \infty
  \end{equation*}
  \end{itemize}
 \end{column}

 \begin{column}{0.45\linewidth}
  \vspace{-4ex}
  \begin{figure}[t]
  \centering
  \includegraphics[width=160mm]{Gr_Kernels2.pdf}
  \end{figure}
 \end{column}

\end{columns}
 


% ---- Figure - Reconstruction -----
\vspace{1ex}
\hrule
\vspace{1ex}
\structure{\bfseries Signal Reconstruction -- Causal versus Non-causal }

\begin{figure}[t]
\centering
\includegraphics[width=290mm]{Gr_Signal3.pdf}
\end{figure}

\vspace{-4ex}

\begin{equation*}
	\widetilde{x}^{\mathrm{ NC}}_{-}(t)
	= \sum^{\infty}_{n=0} \left\langle x, s_{-n} \right\rangle \sigma_{-n}(t)
	\qquad\text{and}\qquad
	\widetilde{x}^{\mathrm{ C}}_{-}(t)
	= \sum^{\infty}_{n=0} \left\langle x, s_{-n} \right\rangle \zeta_{-n}(t)	
\end{equation*}

\begin{itemize}
\item Significant differences close to the border
\item Causal and non-causal reconstruction coincide at the distant past $t \to -\infty$
\end{itemize}

\end{examplelti}
% ===================================================================================================================



\end{column2lti}

\end{columns}
\end{frame}


%%%%%%%%%%%%%%%%%%%%%%%%%%%%%%%%%%%%%%%%%%%%%%%%%%%%%%%%%%%%%%%%%%
%\bibliographystyle{IEEEtran}
%\bibliography{IEEEabrv,../publications,../pub_Pohl,../pub_Books}
%%%%%%%%%%%%%%%%%%%%%%%%%%%%%%%%%%%%%%%%%%%%%%%%%%%%%%%%%%%%%%%%%%

\end{document}

